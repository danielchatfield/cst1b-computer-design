\documentclass{supervision}
\usepackage{course}

\Supervision{5}

\begin{document}

\section*{Lecture 17}

This lecture continues the examination of GPUs and introduces multithreading. At the end, students should understand the basic architecture of a GPU streaming multiprocessor and how this can be programmed in CUDA.

\begin{questions}
    \question
    Why do GPUs rely on multithreading as well as SIMD? What bottleneck does multithreading target?
    \question
    How can a SIMD processor pipeline be modified to allow multithreading?
    \question
    The warp scheduler chooses instructions to execute. What are its criteria for choosing and what other schemes could be implemented?
    \question
    What are the types of memory available within a GPU and what are their relative advantages or disadvantages?
    \question
    What types of application best suit GPUs and how can you program to take advantage of the various forms of parallelism available?
\end{questions}

\section*{Lecture 18}

The final lecture gives an overview of the directions that computer architecture is heading. The aim of this lecture is to give students a perspective on the challenges of the future and an introduction to some of the research that is tackling it.

\begin{questions}
    \question
    Why is energy efficiency the “new fundamental limiter of processor performance”, as Borkar and Chien say?
    \question
    Describe ARM’s big.LITTLE system.
    \begin{parts}
        \part
        What’s the point of having two types of core and when might they be used?
        \part
        Would it be useful to have a third core type and, if so, what would its characteristics be?
    \end{parts}
    \question
    When might it make sense to implement functionality in a specialised accelerator rather than within a general purpose core?
    \question
    What types of application domain might benefit from approximate computing?

\end{questions}

\section*{Case study}

For a processor or \textit{system-on-chip} of your choice (perhaps one you own), try to find the following:

\begin{questions}
    \question
    Its main components
    \question
    Which instruction set(s) and extensions it supports
    \question
    The core's pipeline structure, including numbers and types of functional units and number of available registers
    \question
    Details of the memory hierarchy
    \question
    What it does to reduce the effect of control and data dependencies
    \question
    Its support for parallel execution and memory consistency (if any)
    \question
    What it can do to improve performance and/or power efficiency
    \question
    Its clock speed and fabrication process (e.g. 28/40/65 nm)
    \question
    How it has been tailored to its target market, and the compromises which had to be made
    \question
    How it improved on its predecessor (if any) and how it was improved upon by its successor (if any)
    \question
    Any other noteworthy features or design decisions
    \question
    (Bonus points if you can find or produce a labelled image of the various parts of the chip)
\end{questions}



\end{document}
